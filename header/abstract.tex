\begin{abstract}
When first starting to learn how to program, there is a lot of information to take in. There are a lot of different programming languages, some which have their own rules on how to build and 
execute the developed program. There are tons of different algorithms and different ways a problem can be solved, not to forget all the different terminologies and semantics that exists 
within the field. There is also a great amount of online resources on the Internet, ranging from encyclopaedias (e.g. Wikipedia), tutorial sites (e.g. TutorialsPoint, W3Schools, 
HackThisSite, etc), to online communities (e.g. StackOverflow, CodeProject, etc).
\vspace{0.5em}\newline
When looking for information, searching for an answer, or looking for a solution to a problem, it is not always that easy to come up with a good question (what defines a good question?). 
When learning to program in class, the questions may not come right away. Seeing something explained on the blackboard is quite different from actually understanding and doing it yourself.
Looking for answers online can become quite time-consuming, since the relevance of the results returned by the search engine vary in a large degree. The answer to the question asked may 
not appear until page 10 of the returned results.
\vspace{0.5em}\newline
To help students learning programming, the goal of this thesis is to create a plug-in for the \gls{lms} Open Edx. The plug-in to be created is an Artificial Intelligent Chat Agent (aka. 
a ChatBot), and the students can then ask this Chat Agent questions related to programming. The Chat Agent will use StackOverflow as its Knowledge base, meaning the answers presented to 
the user will be based on the answers posted by users on StackOverflow. By using this Chat Agent, students can ask questions in the same way they would ask a teacher or a classmate, 
instead of having to use keywords when using a search engine. The goal of this thesis is to see if the Chat Agent can aid students that are learning programming, but also do research on 
it (e.g. trying to humanize the Chat Agent (Turing Test)). An example of the experiment would be to have two test groups (A/B testing), where the comparison would be on the grades of 
the students using the Chat Agent vs. those not using it, to see if the Chat Agent had any effect on the students grade.
\end{abstract}