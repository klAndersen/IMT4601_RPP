\chapter{Milestones, deliverables and resources}
\label{chapter5:milestones}

\section{Table of Contents: Master thesis}
\label{chapter5:preliminary_toc}
%% The un-labeled part of the ToC
\begin{itemize}
	\item[] Front page
	\item[] Abstract
	\item[] Acknowledgements
	\item[] Table of contents
	\item[] Glossary
	\item[] Acronyms
\end{itemize}
%% numbered sections of the ToC
\begin{enumerate}[label*=\arabic*.]
	\item Introduction	
	%% Sub-sections within chapter 1: Introductions
	\begin{enumerate}[label*=\arabic*.]
		\item Keywords
		\item Topic covered/Research area\footnote{Although it is called "Topic covered" in this report, it may be more appropriate to call it "Research area" in the Master thesis.}
		\item Problem description
		\item Research questions
		\item Methodology to be used
		\item Justification, Motivation and Benefits
		\item Limitations
		\item Thesis contribution
		\item Thesis structure
	\end{enumerate}
	\item State of the art
	%% Sub-sections within chapter 2: State of the art
	\begin{enumerate}[label*=\arabic*.]
		\item Chat Agents vs. Search Engines
		\item \glsentrylong{ai} (\glsentryshort{ai}) for Chat Agents
		\item Chat Agents for Learning and Education
		\item \glsentrylong{qa} (\glsentryshort{qa}): What defines a good question?
		\item Turing test: Humanizing the AI
	\end{enumerate}
	\item Methodology
	%% Sub-sections within chapter 3: Methodology
	\begin{enumerate}[label*=\arabic*.]
		\item \glsentrylong{hmm} (\glsentryshort{hmm})
		\item \glsentrylong{bn} (\glsentryshort{bn})
		\item A/B Testing
		\item Survey and Interview
		\item Research Design 
	\end{enumerate}
	\item A/B Testing, Surveys and Results
	%% Sub-sections within chapter 4: A/B Testing
	\begin{enumerate}[label*=\arabic*.]
		\item A/B Testing
		\item Interaction with the Chat Agent
		\item Statistical comparison of the students results
	\end{enumerate}
	\item Discussions
	%% Sub-sections within chapter 5: Discussions
	\begin{enumerate}[label*=\arabic*.]
		\item Data and Testing
		\item \glsentrylong{ai} (\glsentryshort{ai}) Methods
		\item Implementation Architecture
		\item Chat Agent vs. Search Engines
	\end{enumerate}
	\item Conclusion/Summary\footnote{Whichever is the appropriate format for the Applied Computer Science Master thesis.}
	%% Sub-sections within chapter 6: Conclusion
	\begin{enumerate}[label*=\arabic*.]
		\item Overview of main results
		\item Further work
	\end{enumerate}
\end{enumerate}
%% bibliography
\begin{itemize}
	\item[] Bibliography
\end{itemize}
%% enumerated list of appendices
\begin{enumerate}[label*=\Alph*.]
	\item Data sets/Statistical Overview
	\item User Survey	
	\item Interview Questionnaire Format
	\item Application Screenshots (Interaction with the Chat Agent)
	\item Miscellaneous information	
\end{enumerate}

\section{Obtaining the desired knowledge}
\label{chapter5:desired_knowledge}
The most important key element in this thesis will be the development of the Chat Agent, since it is the focus of my thesis. Preliminary work has already been done in the course 
IMT5251 Advanced Project Work, where a prototype has been developed. The prototype runs in Open Edx as an XBlock. XBlock runs as an Fragment in Open Edx, allowing developers to 
add their own content which then can be re-used in multiple systems\footnote{For more on XBlock, see 
	 \url{http://edx.readthedocs.org/projects/xblock-tutorial/en/latest/overview/index.html}}. As previously mentioned, the prototype developed is a simplistic version, meaning 
there is no \gls{ai}. The prototype simply takes the users input (the question) and copy/pastes it to search for matching questions on StackOverflow. It then returns the first 
result, where the answer the user sees is either the answer marked as correct, or the top-voted answer (if no answer is marked as correct by original poster). Students from the 
1. and 2. year (who are learning to program) were invited to test the prototype, where I observed them. Afterwards they were asked to fill out a questionnaire (the questionnaire 
is shown in Appendix \ref{appendix:questionnaire}).
\vspace{0.5em}\newline
This means that most of the development needed during the master thesis will be to implement the \gls{ai} and for it to be able to use WordNET for semantics. This would probably take 
a whole calendar month, but it depends on the actual hours invested. If I work between 6-8 hours each day (40+ hours each week), then this should be at least operational at latest 
mid-February. The reason for this extended time is to ensure I have time to setup and test the \gls{ai} properly before allowing students to test them to ensure the collected data is 
valid. When it comes to equipment, I already have most of what I need, since I am already working on the prototype (development is done in Arch Linux). I also have a USB stick 
with Arch Linux installed, so that I can work on my laptop in case something should happen to my Desktop. I have also acquired a student license for PyCharm Professional\footnote{
PyCharm: \url{https://www.jetbrains.com/pycharm/}.} which is valid for 1 year (until 25. November 2016). I am also aware of people that have the required knowledge who I can ask for 
help, such as my supervisor Simon McCallum, Sule Yildirim-Yayilgan, Mariusz Nowostawski and Rune Hjelsvold. 
\vspace{0.5em}\newline
The user testing will of course require students to be willing to partake and test the Chat Agent. The plan is to use A/B testing, so that only part of the students have access to and 
can use this Chat Agent. One thing that may affect the end results are the knowledge level of the users, which means that even if there is an increase in the end results, it may not 
be because of the Chat Agent. The solution to account for this issue is by having the users also grade their own knowledge level, from being novice to expert (e.g. having programmed 
for years). Although there is not a set time limit for required use, if the Chat Agent is not in use, there will not be enough test data. However, the Chat Agent will be available from 
around February to May, and participation may not be that high if there is a weekly requirement for usage. A solution could be to require the participants to at least use the Chat Agent 
for at least 10-20 hours each calendar month. If there then are 20-30 participants, that means the Chat Agent will have a usage of 200-600 hours each month. Which in turn should provide 
a good amount of test data. There will be at least three surveys for the participants, the first when they start using the Chat Agent, the second when testing is halfway and the last 
at the end, to see if there is a correlation between usage, their results and the students own observations.

\section{Produced deliverables}
\label{chapter5:produced_deliverables}

\subsection{Hours needed by me}
\label{chapter5:hours_needed_by_me}
\begin{center}
	\begin{tabular}{| p{4cm} | p{2.5cm} | l |  l | p{4cm} |}
		\hline
		Product & Time (calendar) & Time ('man-hours')\footnotemark & Version \# & Notes  \\
		\hline
		MSc thesis report & January - May & 125-200 Hours & v0.1 - v0.5 & Draft should be presented to supervisor monthly \\ 
		\hline
		1) Chat Agent \newline (\gls{ai} research) & January - \newline February & 50-75 Hours & v0.1 - v0.2 &  Should be available at \newline latest Mid-February for 
		\newline course start-up \\ 
		\hline
		2) Chat Agent \newline (\gls{ai} development) & January -\newline  February & 50-75 Hours & v0.1 - v0.2 & Should be available at \newline latest Mid-February for 
		\newline course start-up \\ 
		\hline
		Meeting w/ supervisor & January - May & 10-20 Hours & v0.1 - v0.5 & Once a week, estimated between 30-60 minutes \\ 
		\hline
		Analyse questions \newline on StackExchange & February - May & 50-100 Hours & v0.2 - v0.5 & One way of doing this \newline would be to use the 
		\newline Chat Agent to e.g. \newline retrieve questions \newline that are closed \\ 
		\hline
		Read various scientific \newline papers & January - May & 50-100 Hours & v0.1 - v0.5 & Read up on the latest \newline published papers to keep up-to-date \\ 
		\hline
		Process student surveys & Jan/Feb, March/April and April/May & 10-30 Hours & v0.2 - v0.4 & Surveys delivered by \newline students participating in \newline the A/B Testing \\ 
		\hline
		Process Student/ \newline Chat Agent interactions & February - May & 60-80 Hours & v0.2 & Should at least try to 
		\newline keep a steady update on \newline the current data on \newline a weekly basis \\ 
		\hline
	\end{tabular}
\end{center}
\footnotetext{Total time spent during set period.} 
	
\subsection{Hours by others}
\label{chapter5:hours_needed_by_others}
\begin{center}
	\begin{tabular}[H]{| p{4cm} | p{2.5cm} | l |  l | p{4cm}|}
		\hline
		Who? & Time (calendar) & Time ('man-hours')\footnotemark & Version \# & Notes  \\
		\hline
		Supervisor & January-May & 10-20 Hours & v0.1 - v0.5 & Once a week, \newline estimated between \newline 30-60 minutes \\ 
		\hline
		Students & February-April/May & 40-80 Hours & v0.2 - v0.5 & Students interaction \newline with the Chat Agent \newline (time per person) \\ 
		\hline
		Students & Jan/Feb, March/April and April/May & 1.5 - 3 hours & v0.2 - v0.4 & Filling out survey/ \newline questionnaire \newline (per student) \\ 
		\hline
	\end{tabular}
\end{center}
\footnotetext{Total time spent during set period.} 
