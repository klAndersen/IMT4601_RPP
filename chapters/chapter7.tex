\chapter{Risk analysis}
\label{chapter7:risk_analysis}
There are two points that can have a major negative impact on my Master thesis. The first is delay of the \gls{ai} (if it is not ready within time), and the 
second is not having enough users to test the Chat Agent. The greatest problem when developing and training \gls{ai} is the time it takes to make it work 
properly, ensuring it can handle invalid data and that it does not use too much time processing and presenting the result to the user. To account for this, 
the goal should not be to have a 100\% perfect \gls{ai}. The most important part in the beginning would be that the \gls{ai} works. This way, the students 
can get started by testing the prototype version with the current \gls{ai}, and then give feedback if something is not working satisfactory. However, if the 
\gls{ai} is too stupid or give to many useless results, this can cause the students to get a negative view and stop using the Chat Agent. 
\vspace{0.5em}\newline
During the prototype testing in IMT5251, although the students liked the concept, they did not find the prototype to be useful or preferable compared to a search engine. 
This is also one of the arguments for why \gls{ai} should be implemented before releasing it to the students. To ensure a to negative bias when using it, 
students will be informed that it is still a prototype under development, so the results presented in the beginning may not be that satisfactory. Which leads 
me back to the issue in regards to having enough testers. As noted in \ref{chapter6:feasibility_study}, a way to get a lot of students to participate is to 
try to get as many course responsibles as possible interested in using it in their courses. Furthermore, should the students drop out, then perhaps those who  
have not tried it might be willing to participate.
\vspace{0.5em}\newline
Since \gls{ai} is a very large field, it can easily become complex and one can also get distracted and lose focus (or get hung up in what one could call "eye candy" 
features). Therefore, I think the best would be in the beginning to have weekly meetings with my supervisor. This way, the supervisor will not only know that work is being done
and progress made, but he can also ensure that the work I have done and are planning do to is what I should focus on. It will also be helpful, because if I get issues which 
I cannot solve (and supervisor is not available at the time), I know that within the next week I have a scheduled meeting and can then focus on something else in the meantime.


\begin{comment}
%% 1/2-2 pages

What can possibly go wrong when you do your project? \\
How do you intend to reduce impact of/solve these problems? \\
Objective: Identify and assess factors that may jeopardize the success of your project \\
- Probability \\
- Consequence \\
Often caused by uncertainties: \\
- Lack of knowledge \\
- Complexity of the project \\
- Inherent randomness \\
Consider the specific risks - not the general ones like getting sick, insufficient feedback from supervisor etc.
\end{comment}