\chapter{Feasibility study}
\label{chapter6:feasibility_study}
As mentioned in \ref{chapter5:desired_knowledge}, the prototype has already been developed, and in addition, two 1.~year Bachelor students have tested it and given feedback. 
Through this feedback I can ensure that the next version is more desirable for the students, and the main focus of the development will be on the \gls{ai}. Aside from 
having a personal interest in \gls{ai}, I have also had two courses in Machine Learning at \gls{guc}. This involved development of both single and hybrid algorithms to 
solve different tasks. E.g.~in the course IMT4641 Computational Forensics I developed a program that analysed Android SQLite databases by using Fuzzy rules and Decision Tree.
Furthermore, I already know what type of hybrid algorithm I will use for the Chat Agents \gls{ai}; \gls{hmm} and \gls{bn}. There is also a lot of research available on 
using these for language processing (and since \gls{hmm} uses states, it can also remember the conversation history).
\vspace{0.5em}\newline
When it comes to the research for questions (and what defines a good question), there is already a lot of data available through the StackExchange community. This means that 
I do not have to rely solely on one of the community pages for my research. E.g.~if StackOverflow is down for maintenance, I can just switch my focus to one of the other 
sites in the community. I also intend to add an editable backend for the teachers in the Chat Agent, so that they can themselves decide which of the StackExchange sites the 
Chat Agent should use. This way, it will not be locked to only one course (or one site), which also means that I can suggests for other professors at \gls{guc} (and potentially 
NTNU) that they could participate in using the Chat Agent. Furthermore, this can also increase the amount of students using the Chat Agent, meaning I can get more data to 
analyse and compare to see if there is any benefit from using the Chat Agent.

\begin{comment}
1/2-3 pages

Why can this project be completed in time?  \\
Resource feasibility \\
- Technical resources \\
- Skills and competencies relevant in solving the problem  \\
- Resources for generating data to be analysed  \\
Schedule feasibility \\
- Size and complexity of tasks, deliverables, and milestones \\
- Headroom for the unexpected \\ \\

E.g. project in Adv. proj. work, comparison to other projects, attempt to answer research question, etc. \\
How to solve the issues/problems in the project
\end{comment}