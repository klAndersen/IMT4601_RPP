\chapter{Related work}
\label{chapter3:related_work}

%% 3-10 pages

\section{Comparison of \glsentrylong{ir} (\glsentryshort{ir}) when using Chat Agents and Search engines}
\label{chapter3:chatbot_vs_search_engine}
%% user experience with chatbots vs. search engines
I could not find anything in relation to this topic on the following sites:
\begin{itemize}
	\item \url{http://dblp.uni-trier.de/}
	\item \url{http://link.springer.com/}
	\item \url{http://ieeexplore.ieee.org/}
	\item \url{http://www.sciencedirect.com/}
	\item \url{http://dl.acm.org/}
\end{itemize}
The following is a list of the keyword searches that were made:
\begin{itemize}
	\item 'chatbot vs search engine'
	\item 'chatbot and search engine'
	\item 'comparison of chatbot and search engine'
	\item 'evaluation of retrieval systems' 
\end{itemize}
The reason for this might be related to the fact that search engines are able to retrieve all sorts of information from numerous documents and web-sites, whereas 
ChatBots are usually made for light conversation or for specific topics and purposes. The research question this is related to (Research question \ref{res_q1}) 
is more on the qualitative side, ie. will the users continue to use the search engine, or would they switch to the Chat Agent?
\vspace{0.5em}\newline
Although it is not an evaluation, in \citet{Crutzen2011} a comparison of the ChatBot Bzz (for Windows Live Messenger) is made against search engines and information lines. 
The goal is to see which is better at answering adolescents' questions related to sex, drugs, and alcohol. The comparison was done by giving the users a questionnaire with 
a 5-point Likert scale. The results showed that the users found the ChatBot to be faster and more anonymous, in addition to being easier to use. Information quantity was 
considered less then both information lines and search engines, and it performed better when it came to conciseness and information quality \citet[p.~517-518]{Crutzen2011}.
\vspace{0.5em}\newline
If one compares this paper to the goal of the Chat Agent I plan to develop, one can see some similarities. Even though search engines can give you numerous results, the quality 
of the results may vary, and there may not be any correlation between what you are looking for and what you find. Whereas with my Chat Agent, the focus is only on the StackExchange 
community, specifically on programming and StackOverflow. Therefore, one could also argue that rather then comparing the Chat Agent against a search engine, perhaps it rather should 
be compared against StackExchange. E.g.~comparing the results based on the question asked in the Chat Agent vs.~the question searched for on the given StackExchange site.
\vspace{0.5em}\newline
There is also a program called FAQ FINDER, which is documented in \citet{Burke1997}. As with my Chat Agent, here users can phrase their questions as they would when asking another 
person, rather then use keywords (as they perhaps would had to when using a search engine\footnote{It should also be noted that I am biased towards it being better to phrase questions 
	to find answers, rather then having to enter a list of keywords.}). 

\section{Chat Agents for Learning and Education}
\label{chapter3:learning_with_chatbots}
There are numerous scientific articles and reports on using ChatBots in education, which are mentioned in many studies 
 \cite{Crutzen2011,Kerly2008,Knill2004,Kowalski2013,Jia2009,Gulenko,Imran2014,Kerly2007,Reed2011,Rossi2011}. Even though the names and definitions varies e.g.~ChatBot, \gls{vt}, 
\gls{ia} and \gls{its}, the main purpose is mostly related to either relieving the teacher of work or to aid the user/students to learn more and acquire new knowledge. In the 
papers by \cite{Kerly2008,Knill2004,Kerly2007} they found that students also wanted the ability to do smalltalk and have off-topic conversations. This would be a useful thing to implement, 
since this can increase the chance that the students will use the Chat Agent, since it will not be restricted to just the curriculum (e.g. being able to ask about the weather 
or just random conversations). There can however also be issues with having too free conversations, since users can attempt to use offensive language, invalid input causing 
the application to hang, spelling/grammatical errors or abuse in some way way (\citet{Kerly2008}). Issues can also come if the knowledge base used is outdated, or is based on resources 
where there is no proper control of who is adding the information (\citet{Knill2004,Imran2014,Reed2011}).
\vspace{0.5em}\newline
Granted, StackExchange is based upon knowledge added by users, and can therefore contain both outdated and invalid answers. However, StackExchange consists mostly of professional 
sites, where both moderators and the members are actively following the posts, both for asked questions and suggested solutions. An example is the ability to mark an answer as 
the correct (can only be done for one answer), and the ability to give votes (up or down) based on the relevance and correctness of an answer. Taking this into consideration, 
bad or invalid answers would then in time be naturally filtered out by having a negative score. Furthermore, users also gets reputation based on the posts they make, be it a 
question, an answer or an comment. This means that if one were to ensure full criticism to retrieved answers, one could set the requirement to only retrieve answers that have 
a score greater then a given value (at least greater than zero), from users with a given reputation score. This does not account for bad language or attempts to perhaps abuse 
the Chat Agent in some way. However, all the conversations are logged, meaning that even without a word filter, one could add a setting for the teacher to retrieve questions 
containing offensive language. One of the issues with word filtering is that words that are not offensive, may be flagged as such (false positives). 
\vspace{0.5em}\newline
The goal is not for the Chat Agent to function as a \gls{vt}, but more of an aiding tool to help students with the more general problems and help them be better at phrasing 
their questions. Not only that, but it can also help the teachers to understand how students learn by looking at the questions they ask the Chat Agent (\citet{Knill2004,Rossi2011}). 
The papers does not list a direct scientific proof that there is a learning improvement by using ChatBots. However, this does not mean that the use of ChatBots cannot have a 
positive impact. As noted in \citet{Kowalski2013}, the quantitative analysis showed no difference between those using and not using a ChatBot, but qualitatively they found that 
the use of a ChatBot was well received, and were open for using it again in the future.

%% usefulness of chatbots in learning
%% does it help students learn
%% partially covers A/B testing, see also next chapter

\section{What is the quality of the results when using \glsentrylong{hmm} (\glsentryshort{hmm}) and \glsentrylong{bn} (\glsentryshort{bn})?}
\label{chapter3:quality_results_hmm_bn}

%% In a given amount of executed queries, how many correct results were presented to the user? 

\section{Passing a limited Turing test}
\label{chapter3:turing_test}

Related papers: \cite{Harnad2000,Livingstone2006,Turing1998}

\section{\glsentrylong{qa} (\glsentryshort{qa}): What defines a good question?}
\label{chapter3:define_good_question}

Related sources: \cite{Stackoverflow.com2015,CommunityWiki2015,Lezina2013,Stackoverflow.com2015a,Stackoverflow.com2015b,Stackoverflow.com2015c,Treude2011}






