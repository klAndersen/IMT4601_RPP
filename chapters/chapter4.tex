\chapter{Choice of methods}
\label{chapter4:choice_of_methods}
%% 2-5 pages

\section{Hyptotheses and variables}
\label{chapter4:hypotheses_variables}
In this section, I will attempt to identify the hypotheses and variables relevant for my Master thesis. The following Hypotheses are based on the research questions, 
and is an attempt to identify the possible outcomes of my research.
\begin{itemize}
	\item H0: The Chat Agent will have no effect on neither the students knowledge, 
	or the students ability to phrase good questions.
	\item H1: The Chat Agent will have no effect on the students knowledge, 
	but the students will be better at phrasing good questions.
	\item H2: The Chat Agent will have an effect on the students knowledge, 
	but not on the students ability to phrase good questions.
	\item H3: The Chat Agent will improve both the knowledge 
	and the students ability to to phrase good questions.
\end{itemize}
The threats to the causality in my thesis is mostly the students maturity. In the beginning they may have little to no knowledge, but as they are coming closer to 
the end of the Spring semester, they will have acquired more understanding and knowledge on the given subjects. Their improvement in asking questions can also be 
affected by their supervisors (e.g. through the iterative process of asking questions, they learn to be more specific when asking for help). Previous knowledge from 
before the Bachelor started can also have an impact. Students with background in programming, be it self-taught or through work may know the basics, but its first 
at the end of the course they can put all the pieces together (the third variable problem) (\citet{Tafliovich2013}). 

\section{Survey and Interview}
\label{chapter4:survey_and_interview}
To identify such underlying issues, all participants will be given surveys. The goal of this survey is to try to find out if there is a correlation between the use of the 
Chat Agent and the knowledge acquired at the end of the semester. The survey will ask them questions related to their current knowledge level, if they have any previous 
experience with programming and their field of interest\footnote{Since interest level can have an effect on the amount of time the student spends acquiring new knowledge 
	for that field.}. The basis will be the Likert 5-point scale, and it would be an improvement of the questionnaire used in IMT5251 (shown in Appendix \ref{appendix:questionnaire}, 
p.~\pageref{appendix:questionnaire}). It is also necessary to find out in what way the different students learn, e.g. by using Fleming's VARK 
 questionnaire\footnote{\url{http://vark-learn.com/the-vark-questionnaire/?p=questionnaire}} (used in \citet[p.~152]{Kowalski2013} and \citet{Sarabdeen2013}). 
The results of the VARK questionnaire can be used to see if there is a correlation between those that are of the type read/write and their grades at the end.
\newpage\noindent
A combination of semi-structured interviews and observations may also prove beneficial (\citet{Fincher2011}). This can be achieved by attending some of the classes, and 
observe the students when they use the Chat Agent (both for problem-solving and during lectures). Students could then also be randomly selected for follow up interviews 
based on the observation. The basis of the interview questions would of course be the same as the questions in the survey. This can then be used to see if there are any 
issues or variables which the survey does not cover. Examples of follow up questions could be explanations ("Why did you do X while using the Chat Agent?"), how often 
they use the Chat Agent, elaboration on a given topic, or feedback and suggestions for improvement ("What would make you want to use the Chat Agent more?"). 

\section{Random Controlled Trial (RCT) as an experimental method}
\label{chapter4:rtc}
The students will be dived randomly into two groups, where the first would be in the experimental group, and the rest in the control group. The difference for these two 
groups is that the control group will only have access to the course content, and the experimental group will also have access to the Chat Agent. The groups will not know 
what is being tested, if it is the use of edX or if it is the Chat Agent. 
\vspace{0.5em}\newline
In the paper by \citet{Bezakova2014}, RTC was used to see if an open-ended study would increase the students grade. They separated the lab sessions for the experimental 
and control group to minimize bias from attributes. They note that in an ideal setting, the participants would not know which group they belong to. The same goes for 
my thesis, because there is a possibility only students at \gls{guc} will attend. This makes it hard to achieve blindness, since students will talk to each other, 
and most likely will notice that others have access to something they do not (ie. the Chat Agent).

\section{\glsentrylong{qa} (\glsentryshort{qa}) model}
\label{chapter4:qa_model}
\citet{Mishra2015} presents question posting as a learning mechanism. It consists of two phases; Instruction phase (initial lecture) and Question Posing (QP) phase 
(clarification or acquire new knowledge based on the seed). This question posting can then be used to create a qualitative analysis based on the following questions:
\begin{itemize}
	\item Starting from the first question (seed), is there an improvement in the questions asked 
	(e.g. amount of questions asked until reaching an answer)?
	\item Does the students question(s) follow a pattern (e.g. iterative, copying teacher, etc.)?
	\item Upon reaching an answer, does the student look for more information\footnote{
		Presented answers will have varying length, and long answers will be shortened with a "Read More?" option.
		In addition, students will be asked to verify whether or not the currently presented answer answered their question.
		What could then be done is to check if attempts to acquire new knowledge is made, by comparing the topic of the current 
		and previous questions after an answer is marked as correct (by also looking at the answers that were read).
		}?
\end{itemize}
% This can also be used to see how precise and accurate the Chat Agents \gls{ai} is.

\section{Quantitative comparison of the students results}
\label{chapter4:quantitative_comparison}
How can we see if there is any notable difference in the students results? This can be achieved by not only comparing the results of the experimental and control group, but 
also the results of last years students. The analysis would be based on either Analysis of Variation (ANOVA) or Dunnett's test\footnote{Dunnett's test is effective 
	when the sample sizes are small, the separate populations are not normally distributed, and their variances are not equal (as determined by separate tests) 
	(\citet[p.~3]{Simon2011}).}. Dunnett's test was used in \citet{Simon2011} to compare the results of the current students and those from the previous year. 
\citet{Tafliovich2013} combined the quantitative results (the grades) previous experience to see if this had any effect on their grades (they used Wilcoxon rank-sum test 
to compare the median of students with and without previous experience).
\vspace{0.5em}\newline
Students results can also be affected by what type of learning theory the teacher uses. \citet{Kinnunen2007} found in their phenomenographic\footnote{"Phenomenography is a 
	research approach that is designed 	to elicit the variation in ways of experiencing things on a communal level. The roots of the approach are strongly
	connected to empirical studies of learning" (\citet[p.~3]{Kinnunen2007}).} study five different categories\footnote{The five categories are: \\
	Subject: Instructor only presents their material. This is how the students learn. \\
	Intrinsic: Students quality is based on their own abilities, and is not something the instructor can change. \\
	Previous experience: The knowledge the student had before starting the course is the reson for their success. \\
	Attitude/Behaviour: The students investment in the course (e.g. workload, hours, etc.).  \\
	Developmental: Instructor uses different strategies to encourage learning (e.g. understanding the topic, new ways of thinking, skill usage, etc.)
	(\citet[p.~5-7]{Kinnunen2007})
	} on how instructors understands why students fail and succeed. 
It could therefore be interesting to also interview the teachers of each course to find out how they teach their course, what their definition of a successful student is, 
and whether or not they think it is important that they can ask good questions. This could help identify third variable problems in relation to asking good questions. If 
the teacher is not in the Developmental category, it is a high probability that the teacher does not focus on improving question quality. If this is true for all teachers 
for a given experimental group, it could imply that for this group, the Chat Agent had an impact on their question quality.

\section{Short summary of the process}
\label{chapter4:the_process}
\begin{enumerate}
	\item Develop the \gls{ai} for the Chat Agent
	\item Randomly distribute students into control and experimental group
	\item Have all students complete an introductory, mid-term and end-term survey
	\item Observe students during class and interview students after observation
	\item Analysing questions found on StackOverflow (e.g. \citet{Movshovitz-Attias2013})
	\item Analysing the questions asked by students by looking the quantity 
	(amount of questions asked, amount of answers read, amount of answers marked as correct)
	\item Interview teachers to draw conclusions on their learning style
	\item Analyse the students results at the end by looking at variables 
	(e.g. grades, previous experience, teachers learning style, interests, questions asked)
\end{enumerate}
