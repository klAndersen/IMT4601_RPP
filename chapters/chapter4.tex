\chapter{Choice of methods}
\label{chapter4:choice_of_methods}
%% 2-5 pages

\section{Hyptotheses and variables}
\label{chapter4:hypotheses_variables}
In this section, I will attempt to identify the hypotheses and variables relevant for my Master thesis. The following Hypotheses are based on the research questions, 
and is an attempt to identify the possible outcomes of my research.
\begin{itemize}
	\item H0: The Chat Agent will have no effect on neither the students knowledge, 
	or the students ability to phrase good questions.
	\item H1: The Chat Agent will have no effect on the students knowledge, 
	but the students will be better at phrasing good questions.
	\item H2: The Chat Agent will have an effect on the students knowledge, 
	but not on the students ability to phrase good questions.
	\item H3: The Chat Agent will improve both the knowledge 
	and the students ability to to phrase good questions.
\end{itemize}
The threats to the causality in my thesis is mostly the students maturity. In the beginning they may have little to no knowledge, but as they are coming closer to 
the end of the Spring semester, they will have acquired more understanding and knowledge on the given subjects. Their improvement in asking questions can also be 
affected by their supervisors (e.g. through the iterative process of asking questions, they learn to be more specific when asking for help). Previous knowledge from 
before the Bachelor started can also have an impact. Students with background in programming, be it self-taught or through work may know the basics, but its first 
at the end of the course they can put all the pieces together (the third variable problem). 

\section{Survey and Interview}
\label{chapter4:survey_and_interview}
To identify such underlying issues, everyone in the classes that participate will be given surveys to fill out during the thesis. The goal of this survey is to try 
to find out if there is a correlation between the use of the Chat Agent and the knowledge acquired at the end of the semester. The survey will ask them questions 
related to their current knowledge level and what experience, if any, they have from before. It is also necessary to find out in what way the different students learn, 
e.g. by using Fleming's VARK questionnaire\footnote{\url{http://vark-learn.com/the-vark-questionnaire/?p=questionnaire}} (used in \citet[p.~152]{Kowalski2013} and 
\citet{Sarabdeen2013}). This way, 
I can also try to see if there is a correlation between those that are of the type read/write and their grades at the end.
\vspace{0.5em}\newline
It would also be necessary to conduct interviews with participants, to see if there are any issues or variables that the survey has not picked up. \\
\emph{TODO: write more here\ldots}

\section{A/B testing as an experimental method}
\label{chapter4:ab_testing_experimental}
The students will be dived randomly into two groups; an experimental and a control group. The control group will follow the course the same way the last years students 
did. The same goes for the experimental group, but in addition they will have access to the Chat Agent. This means that the experiment will be a semi-quasi experiment 
(\citet[p.~226-248]{Leedy2012} and \citet[p.~114-115]{Ringdal2007}).

\section{\glsentrylong{qa} (\glsentryshort{qa}) model}
\label{chapter4:qa_model}
The students ability to ask better questions can be analysed by looking at the questions they ask the Chat Agent. Learning to ask better questions is an iterative 
process, and by comparing the questions asked in the beginning and the end, it might be possible to see if there is a qualitative improvement in their questions.
This can also be done quantitatively, by comparing the amount of questions asked before marking the received answer as correct. 
\vspace{0.5em}\newline
Since answers will be of various length, long answers will be shortened with a "Read More?". When the user then clicks the "Read More?", the value that this answer 
was read will be stored in the database. This can then be used as a measurement to see out of all answers retrieved for a given question, how many were of interest 
to the user (in addition to looking at the answer marked as correct by the Chat Agent user). This can also be used to see how precise and accurate the Chat Agents 
\gls{ai} is.

\section{Quantitative comparison of the students results}
\label{chapter4:quantitative_comparison}
How can we see if there is any notable difference in the students results? This can be done by comparing the results against both the students not participating, but 
also against the students from the previous year. This could be done by using either Analysis of Variation (ANOVA) or Dunnett's test\footnote{	Dunnett's test is effective 
	when the sample sizes are small, the separate populations are not normally distributed, and their variances are not equal (as determined by separate tests) 
	(\citet[p.~3]{Simon2011}).}. Dunnett's test was used in \citet{Simon2011} to compare the results of the current students and those from the previous year.

